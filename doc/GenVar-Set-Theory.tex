\chapter{General-Variable Set Theory}\label{chp:genvar-set-theory}

\section{Introduction}

This chapter takes the ideas from the chapter on matching Features
\ref{chp:match-feat}
and develops a sound theory of set operations over sets containing
both standard and list variables (a.k.a. ``general variables'').

We will start with a very general abstraction in which we ignore the
nuances of variable classes and temporality.

\section{Abstract GenVars}

\subsection{Variable Definitions}

Standard variables ($StdVar$) take on values from some observation space.
Here these are written using single lowercase letters, 
possibly with simple subscripts:
%
$$a,b,c_1,d_i,\dots  ~:~  StdVar$$
%
List variables ($LstVar$) take on values that denote sets of standard variables.
Their \emph{base} form is written 
using a lowercase letter followed by the dollar symbol ($\lst{}$),
possibly with simple subscripts:
%
$$\lst z,\lst y,\lst x_1,\lst v_i,\dots ~:~ LstVar$$
%
General variables can be Standard and/or List variables
($GenVar = StdVar \uplus LstVar$).
We assume two predicates $isStd$ and $isLst$ that distinguish these.
When the distinction is not important,
they are written as single lowercase Greek letters, 
possibly with simple subscripts:
%
$$\alpha,\beta,\gamma_1,\delta_i,\dots ~:~ GenVar$$
%
List variables have an \emph{extended} form where they have an associated list
of general variables, each in base form only,
that denote the removal of these variables from the top-level base form.
%
$$\lst z\less{\beta_1,\dots,\beta_n} ~:~ LstVar$$
%
We also require that no $\beta$ can have $\lst z$ as its base form.

\subsection{GenVar Sets}

We want to reason about sets of general variables 
($\setof{\alpha,\beta,\gamma_2,\delta_i,\dots} ~:~ GVSet$)
associated with the usual set-theoretic operations such as
membership ($\in$),
union ($\cup$),
intersection ($\cap$),
and removal ($\setminus$).
%
$$GVSet ~=~ \power GenVar$$
%
We expect all the usual laws of such operators to hold.
Variables denoting these sets are written as single uppercase letters, 
possibly with simple subscripts:
%
$$A,B,C_1,D_i,\dots ~:~ GVSet$$

\subsection{Simplifying Extended List-Variables}

If we have a set of list variables, with some sharing the same base form,
but with differing removal lists, these can be simplified as follows:
\begin{eqnarray*}
   \lst x \less\emptyset &=& \lst x
\\ \lst x \less A, \lst x \less B &=& \lst x \less{A\cap B}
\\ \lst x,\lst x \less B &=& \lst x
\end{eqnarray*}
This should always be done, 
so that in any set, 
the base form of any given list variable appears at most once.

\subsection{GenVar Interpretation}

A GenVar Set ($GVSet$) is intended to ultimately define
a specific set of \emph{standard} variables.
Any standard variable explicitly denotes itself,
while list variables need an \emph{interpretation} 
that maps them to a set of standard variables.
In general we do not require an interpretation to have entries
for all list variables present, 
nor do we require any such entry to provide just standard variables.
However, 
we do rule out cyclic interpretations where list variables are mapped
to variable sets that contain themselves.
This flexibility regarding interpretations is needed 
because we have many general laws 
that are designed to apply to a wide range of theories,
each of which will have its own specific collection of standard variables.

We define an interpretation ($Intp$) as a partial mapping from list variables
to sets of general variables:
%
\def\intp{\mathcal I}
$$\intp ~:~ Intp ~=~ LstVar \pfun GVSet$$
%
A \emph{ground} interpretation is one were all sets in its range
contain only standard variables.

An interpretation is \emph{complete} w.r.t a set 
when every list variable mentioned in that set 
has an mapping in the interpretation.

An interpretation is \emph{full} w.r.t a set 
when it is ground and complete for that set.

Once we have interpretations,
we can add extra laws, e.g.:
\begin{eqnarray*}
   y,\lst x \less{A \uplus \setof{y}} 
   &=& \lst x \less A
   \quad \text{provided } y \in \intp(\lst x)
\end{eqnarray*}

\subsection{GenVar Set Evaluation}

We start the definition of set evaluation w.r.t. an interpretation,
by focussing on a single variable and doing a lookup ($gvlkp$):
\begin{eqnarray}
   gvlkp                   &  :  & Intp \fun GenVar \fun GVSet
\\ gvlkp_\intp(x)      &\defs& \setof{x}
\\ gvlkp_\intp(\lst x)
   &\defs& 
   \setof{\lst x} \cond{\lst x \notin \dom\intp} \intp(\lst x)
\\ gvlkp_I(\lst x \less{\beta_1,\dots,\beta_n})
   &\defs& 
   tidyup
     (  gvlkp_\intp(\lst x)
     ,  \bigcup map(gvlkp_\intp\setof{\beta_1,\dots,\beta_n}) )
\end{eqnarray}
Note that $\intp(\lst xs)$ can contain extended list-vars,
including $\lst x\less B$.
The $tidyup$ function returns a set of general variables,
those returned by the first $gvlkp$ call,
with extended annotations drawn from the mapped $gvlkp$ calls,
simplified as far as possible.
\begin{eqnarray}
   tidyup &:& GVSet \fun GVSet \fun GVSet
\\ tidyup(\emptyset,\_) &\defs& \emptyset
\\ tidyup(\setof{\nu},D) &\defs& simpless(\nu,D)
\\ tidyup((U \cup V),D) 
   &\defs& tidyup(U,D) \cup tidyup(U,D)
\end{eqnarray}
The last line holds true regardless of how the input set contents ($W$)
are distributed between $U$ and $V$, as long as $W=U \cup V$.
Ideally both $U \subset W$ and $V \subset W$ should hold too!
E.g.:
%
$$
  tidyup(W,D) 
  =
  simpless(\nu,D)
  \cup 
  tidyup(W\setminus\setof{\nu},D)
  , \nu \in W
$$
With $simpless(\nu,D)$ we are going construct $\nu\less{D}$,
simplifying where we can, noting that $\nu$ itself may have an extended form.
We will return either such a variable, or an empty set.
Note that $\nu$ can be either standard or a list variable.
\begin{eqnarray}
   simpless &:& GenVar \fun GVSet \fun GVSet
\\ simpless(x,D) &\defs& \emptyset \cond{x \in D} \setof{x}
\\ simpless(\lst x,D)
   &\defs&
   \emptyset
   \cond{\lst x \in D}
   ( \setof{\lst x\less\beta}
     \cond{ \beta \in D}
     \setof{\lst x}
   )
\end{eqnarray}
The case $(\lst x\less D,D')$ is complicated.


We now want to extend this to 
evaluating a GenVar set w.r.t. to some interpretation.
We start with a single step that looks up all the list variables,
and adds those results in with all the standard variables ($intstep$):
\begin{eqnarray}
   intstep                  &  :  & Intp \fun GVSet \fun GVSet
\\ intstep_\intp(\emptyset) &\defs& \emptyset
\\ intstep_\intp(\setof{\nu}) &\defs& gvlkp_I\sem{\nu}
\\ intstep_\intp(U \cup V)    &  =  & intstep_I(U) \cup intstep_I(V)
\end{eqnarray}
Finally, we repeatedly apply $intstep$ 
until the result stops changing:
\begin{eqnarray}
   vseval          &  :  & Intp \fun GVSet \fun GVSet
\\ vseval_\intp(W) 
   &\defs& 
   \LET~W' = intstep_I(W)
   ~\IN~ W \cond{ W=W' } vseval_I(W')
\end{eqnarray}
If $I$ is complete, then this results in a set containing only standard variables.

