\chapter{General-Variable Set Theory}\label{chp:genvar-set-theory}

\section{Introduction}

This chapter takes the ideas from the chapter on matching Features
\ref{chp:match-feat}
and develops a sound theory of set operations over sets containing
both standard and list variables (a.k.a. ``general variables'').

We will start with a very general abstraction in which we ignore the
nuances of variable classes and temporality.

\section{Abstract GenVars}

\subsection{Variable Definitions}

Standard variables ($StdVar$) take on values from some observation space.
Here these are written using single lowercase letters, 
possibly with simple subscripts:

$$a,b,c_1,d_i,\dots  ~:~  StdVar$$

List variables ($LstVar$) take on values that denote sets of standard variables.
Their \emph{base} form is written 
using a lowercase letter followed by the dollar symbol ($\lst{}$),
possibly with simple subscripts:

$$\lst z,\lst y,\lst x_1,\lst v_i,\dots ~:~ LstVar$$

General variables can be Standard and/or List variables
($GenVar = StdVar \uplus LstVar$).
When the distinction is not important,
they are written as single lowercase Greek letters, 
possibly with simple subscripts:

$$\alpha,\beta,\gamma_1,\delta_i,\dots ~:~ GenVar$$

List variables have an \emph{extended} form where they have an associated list
of general variables, 
which denote the removal of (standard) variables from the base form.

$$\lst z\less{\beta_1,\dots,\beta_n} ~:~ LstVar$$

\subsection{GenVar Sets}

We want to reason about sets of general variables
($\setof{\alpha,\beta,\gamma_2,\delta_i,\dots}$).
associated with the usual set-theoretic operations such as
membership ($\in$),
union ($\cup$),
intersection ($\cap$),
and removal ($\setminus$).
We expect all the usual laws of such operators to hold.
Variables denoting these sets are written as single uppercase letters, 
possibly with simple subscripts:

$$A,B,C_1,D_i,\dots ~:~ \power GenVar$$

\subsection{GenVar Interpretation}

A GenVar Set defines a set of standard variables.
Any standard variable explicitly denotes itself,
while list variables need an \emph{interpretation} 
that maps them to a set of standard variables.
In general we do not require an interpretation to have entries
for all list variables present, 
nor do we require any such entry to provide just standard variables.
However, 
we do rule out cyclic interpretations where list variables are mapped
to variable sets that contain themselves.
This flexibility regarding interpretations is needed 
because we have many general laws 
that are designed to apply to a wide range of theories,
each of which will have its own specific collection of standard variables.

We define an interpretation ($Intp$) as a partial mapping from list variables
to sets of general variables:

\def\intp{\mathcal I}
$$\intp ~:~ Intp ~=~ LstVar \pfun \power GenVar $$

A \emph{ground} interpretation is one were all GenVar sets in its range
only contain standard variables.

An interpretation is \emph{complete} w.r.t a GenVar set 
when every list variable mentioned in that set 
has an mapping in the interpretation.

An interpretation is \emph{full} w.r.t a GenVar set 
when it is ground and complete for that set.

\subsection{GenVar Set Evaluation}

We start the definition of set evaluation w.r.t. an interpretation,
by focussing on a single list variable and doing a single lookup ($lvstep$):
\begin{eqnarray}
   lvstep &:& Intp \fun LstVar \fun \power GenVar
\\ lvstep_\intp\sem{\lst x}
   &\defs& 
   \setof{\lst x} \cond{\lst x \notin \dom\intp} \intp(\lst x)
\end{eqnarray}

TO COME, full instantiation.

